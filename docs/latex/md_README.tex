This repo contains a high-\/performance implementation of Bandit\+P\+AM from \href{https://arxiv.org/abs/2006.06856}{\tt https\+://arxiv.\+org/abs/2006.\+06856}. The code can be called directly from Python or C++.

If you use this software, please cite\+:

Mo Tiwari, Martin Jinye Zhang, James Mayclin, Sebastian Thrun, Chris Piech, Ilan Shomorony. \char`\"{}\+Bandit-\/\+P\+A\+M\+: Almost Linear Time k-\/medoids Clustering via Multi-\/\+Armed Bandits\char`\"{} Neural Information Processing Systems (Neur\+I\+PS) 2020.


\begin{DoxyCode}
@inproceedings\{BanditPAM,
  title=\{Bandit-PAM: Almost Linear Time k-medoids Clustering via Multi-Armed Bandits\},
  author=\{Tiwari, Mo and Zhang, Martin J and Mayclin, James and Thrun, Sebastian and Piech, Chris and
       Shomorony, Ilan\},
  booktitle=\{Advances in Neural Information Processing Systems\},
  pages=\{368--374\}, #TODO: Fix this
  year=\{2020\}
\}
\end{DoxyCode}


\section*{Python Quickstart}

\subsection*{Install the repo and its dependencies\+:}


\begin{DoxyCode}
/BanditPAM/: pip install -r requirements.txt
/BanditPAM/: sudo pip install .
\end{DoxyCode}


\subsubsection*{Example 1\+: Synthetic data from a Gaussian Mixture Model}


\begin{DoxyCode}
from BanditPAM import KMedoids
import numpy as np
import matplotlib.pyplot as plt

# Generate data from a Gaussian Mixture Model with the given means:
np.random.seed(0)
n\_per\_cluster = 40
means = np.array([[0,0], [-5,5], [5,5]])
X = np.vstack([np.random.randn(n\_per\_cluster, 2) + mu for mu in means])

# Fit the data with BanditPAM:
kmed = KMedoids(n\_medoids = 3, algorithm = "BanditPAM")
kmed.fit(X, 'L2')

# Visualize the data and the medoids:
for p\_idx, point in enumerate(X):
    if p\_idx in map(int, kmed.final\_medoids):
        plt.scatter(X[p\_idx, 0], X[p\_idx, 1], color='red', s = 40)
    else:
        plt.scatter(X[p\_idx, 0], X[p\_idx, 1], color='blue', s = 10)
plt.show()
\end{DoxyCode}




\subsubsection*{Example 2\+: M\+N\+I\+ST and its medoids visualized via t-\/\+S\+NE}


\begin{DoxyCode}
from BanditPAM import KMedoids
import numpy as np
import pandas as pd
import matplotlib.pyplot as plt
from sklearn.manifold import TSNE

# Load the 1000-point subset of MNIST and calculate its t-SNE embeddings for visualization:
X = pd.read\_csv('data/MNIST-1k.csv', sep=' ', header=None).to\_numpy()
X\_tsne = TSNE(n\_components = 2).fit\_transform(X)

# Fit the data with BanditPAM:
kmed = KMedoids(n\_medoids = 10, algorithm = "BanditPAM")
kmed.fit(X, 'L2')

# Visualize the data and the medoids via t-SNE:
for p\_idx, point in enumerate(X):
    if p\_idx in map(int, kmed.final\_medoids):
        plt.scatter(X\_tsne[p\_idx, 0], X\_tsne[p\_idx, 1], color='red', s = 40)
    else:
        plt.scatter(X\_tsne[p\_idx, 0], X\_tsne[p\_idx, 1], color='blue', s = 5)
plt.show()
\end{DoxyCode}


\subsection*{Building the C++ executable from source}

Please note that it is not necessary to build the C++ executable from source to use the Python code above. However, if you would like to use the C++ executable directly, follow the instructions below.

\subsubsection*{Option 1\+: Building with Docker}

We highly recommend building using Docker. Once you have Docker installed and the Docker Daemon is running, run the following commands\+:


\begin{DoxyCode}
/BanditPAM$ chmod +x env\_setup.sh
/BanditPAM$ ./env\_setup.sh
/BanditPAM$ ./run\_docker.sh
\end{DoxyCode}


which will start a Docker instance with the necessary dependencies. Then\+:


\begin{DoxyCode}
/BanditPAM$ mkdir build && cd build
/BanditPAM/build$ cmake .. && make
\end{DoxyCode}


This will create an executable named {\ttfamily Bandit\+P\+AM} in {\ttfamily Bandit\+P\+A\+M/build/src}.

\subsubsection*{Option 2\+: Installing Requirements and Building Directly}

Building this repository requires three external requirements\+:
\begin{DoxyItemize}
\item Cmake $>$= 3.\+17
\item Armadillo $>$= 9.\+7, \href{http://arma.sourceforge.net/download.html}{\tt http\+://arma.\+sourceforge.\+net/download.\+html}
\item Open\+MP $>$= 2.\+5, \href{https://www.openmp.org/resources/openmp-compilers-tools/}{\tt https\+://www.\+openmp.\+org/resources/openmp-\/compilers-\/tools/} (Open\+MP is supported by default on most Linux platforms, and can be downloaded through homebrew on macs.)
\item C\+A\+R\+MA $>$= 0.\+3.\+0, \href{https://github.com/RUrlus/carma}{\tt https\+://github.\+com/\+R\+Urlus/carma}
\end{DoxyItemize}

Ensure all the requirements above are installed and then run\+:


\begin{DoxyCode}
/BanditPAM$ mkdir build && cd build
/BanditPAM/build$ cmake .. && make
\end{DoxyCode}


This will create an executable named {\ttfamily Bandit\+P\+AM} in {\ttfamily Bandit\+P\+A\+M/build/src}.

\subsection*{Usage}

Once the executable has been built, it can be invoked with\+: 
\begin{DoxyCode}
/BanditPAM/build/src/BanditPAM -f [path/to/input.csv] -k [number of clusters] -v [verbosity level]
\end{DoxyCode}



\begin{DoxyItemize}
\item {\ttfamily -\/f} is mandatory and specifies the path to the dataset
\item {\ttfamily -\/k} is mandatory and specifies the number of clusters with which to fit the data
\item {\ttfamily -\/v} is optional and specifies the verbosity level.
\end{DoxyItemize}

For example, if you ran {\ttfamily ./env\+\_\+setup.sh} and downloaded the M\+N\+I\+ST dataset, you could run\+:


\begin{DoxyCode}
/BanditPAM/build/src/BanditPAM -f ../data/MNIST-1k.csv -k 10
\end{DoxyCode}
 